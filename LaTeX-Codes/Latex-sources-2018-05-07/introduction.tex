\IEEEPARstart{I}{mage} segmentation has been a hot research direction for the past decades. Its main task is to accurately mark the desired regions in the image \cite{Introduction:pal1993review}. And various methods have been proposed to solve this problem. The rise of deep learning has refreshed the research methods of many problems and has become a watershed of "traditional" methods and "deep" methods \cite{Introduction:lecun2015deep}. Similarly, image segmentation is also divided into traditional methods and deep learning methods.

Most of the traditional methods for image segmentation are based on the global or local statistics information of a single image. The segmentation process for the target regions is mainly based on the pre-defined statistical assumptions. Threshold-based methods determine the threshold grayscale based on the global/local grayscale histogram of image, and the threshold can be determined adaptively \cite{Introduction:traditional:threshold:sezgin2004survey}. Edge-based methods are based on "the regions in the image uniquely determine the edges". Firstly, the edges is detected by the edges detection methods, and then the segmentation regions is determined by the edges \cite{Introduction:traditional:edge:senthilkumaran2009edge}. Region-based methods obtain the final segmentation regions by dividing and merging similar regions \cite{Introduction:traditional:watershed:nguyen2003watersnakes} \cite{LevelSet:superpixels:achanta2012slic}. Graph-based methods represent image by graph structure, and divide the graph structure to achieve the purpose of image segmentation \cite{Introduction:traditional:graph:felzenszwalb2004efficient}. Mean Shift method maps all points to high-dim feature space, and divides the regions by mean shift clustering \cite{Introduction:traditional:meanShift:comaniciu2002mean}. Active contour-based methods model the target contour explicitly \cite{Introduction:traditional:snakes:kass1988snakes} or implicitly \cite{LevelSet:chan2001active}. From the point of now, these traditional methods can be viewed as unsupervised methods. They are based on artificially defined patterns rather than patterns learned from labeled segmentation results.

Different from the traditional methods, the deep learning methods are more inclined to find the patterns in images through the training set training. Deep neural networks (DNN) have powerful ability to represent high-level features, so the image segmentation task in DNN is extended to semantic segmentation to segment regions with complex high-level semantic information. Deep Learning methods for image segmentation predict the category of each pixel in image. Fully Convolutional Networks (FCN) replace fully connected layers in the network with convolutional layers to accomplish this dense prediction \cite{FCN-original:long2015fully}. Subsequently, most of the deep learning methods for image segmentation are based on this idea \cite{Introduction:FCN:ronneberger2015u} \cite{Introduction:FCN:badrinarayanan2017segnet}. It is worth mentioning that the probabilistic models such as Conditional Random Fields (CRFs) have been added to FCNs to solve the problems of noisy and imprecise at boundaries \cite{Introduction:FCN:chen2018deeplab} \cite{FCN:CRF:zheng2015conditional}.

Deep convolutional networks can effectively extract the hidden patterns in images, and can learn realistic image priors from a large number of example images \cite{Introduction:FCN:prior:UlyanovVL17}. So many methods use the result of deep learning as a probabilistic prior. Many methods combining level set and deep learning are proposed for image segmentation. Most of them use the prior obtained by deep learning to initialize the surface $u_0$ of Level Set and as shape prior in the iterative process \cite{Introduction:deep:LevelSet:hu2017deep} \cite{Introduction:deep:LevelSet:tang2017deep}. These methods rely too much on prior knowledge generated by deep leaning, but such priors are also rough and imprecise. And they don't consider the inherent shape prior of the target.

In this paper, based on the advantages of combining deep learning with level set methods, a shape prior representing the intrinsic shape of the target is added. And the shape prior is adjusted with affine transformation to fit a specific image by Global Affine Transformation (GAT). Finally, we combine the information of original image, the probability map and the corrected shape prior together with Level Set method to obtain the segmentation results. And the Portrait data set \cite{FCN:segmentation:shen2016automatic} is chosen for experiments, and some samples are shown in Fig. \ref{fig: Some images and ground truth of Portrait data set}. Because the task of the data set is relatively simple and there only one salient target in per image. It is convenient to make preliminary verification of the proposed method.

The rest of this paper is organized as follows. In Section \ref{sec:Propsed Work}, the detail of the proposed method is described and how to integrate the original image, the probability map and the corrected shape prior is introduced in detail. The Portrait data set and the experimental results are shown in Section \ref{sec:Experiments and Results}. In section \ref{sec:Discussion}, some ideas of image segmentation and the method of combining deep learning with Level Set method. Finally, the conclusion and future work are given in Section \ref{sec:Conclusion and Future Work}.
